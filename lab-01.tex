% Options for packages loaded elsewhere
% Options for packages loaded elsewhere
\PassOptionsToPackage{unicode}{hyperref}
\PassOptionsToPackage{hyphens}{url}
\PassOptionsToPackage{dvipsnames,svgnames,x11names}{xcolor}
%
\documentclass[
  letterpaper,
  DIV=11,
  numbers=noendperiod]{scrartcl}
\usepackage{xcolor}
\usepackage{amsmath,amssymb}
\setcounter{secnumdepth}{-\maxdimen} % remove section numbering
\usepackage{iftex}
\ifPDFTeX
  \usepackage[T1]{fontenc}
  \usepackage[utf8]{inputenc}
  \usepackage{textcomp} % provide euro and other symbols
\else % if luatex or xetex
  \usepackage{unicode-math} % this also loads fontspec
  \defaultfontfeatures{Scale=MatchLowercase}
  \defaultfontfeatures[\rmfamily]{Ligatures=TeX,Scale=1}
\fi
\usepackage{lmodern}
\ifPDFTeX\else
  % xetex/luatex font selection
\fi
% Use upquote if available, for straight quotes in verbatim environments
\IfFileExists{upquote.sty}{\usepackage{upquote}}{}
\IfFileExists{microtype.sty}{% use microtype if available
  \usepackage[]{microtype}
  \UseMicrotypeSet[protrusion]{basicmath} % disable protrusion for tt fonts
}{}
\makeatletter
\@ifundefined{KOMAClassName}{% if non-KOMA class
  \IfFileExists{parskip.sty}{%
    \usepackage{parskip}
  }{% else
    \setlength{\parindent}{0pt}
    \setlength{\parskip}{6pt plus 2pt minus 1pt}}
}{% if KOMA class
  \KOMAoptions{parskip=half}}
\makeatother
% Make \paragraph and \subparagraph free-standing
\makeatletter
\ifx\paragraph\undefined\else
  \let\oldparagraph\paragraph
  \renewcommand{\paragraph}{
    \@ifstar
      \xxxParagraphStar
      \xxxParagraphNoStar
  }
  \newcommand{\xxxParagraphStar}[1]{\oldparagraph*{#1}\mbox{}}
  \newcommand{\xxxParagraphNoStar}[1]{\oldparagraph{#1}\mbox{}}
\fi
\ifx\subparagraph\undefined\else
  \let\oldsubparagraph\subparagraph
  \renewcommand{\subparagraph}{
    \@ifstar
      \xxxSubParagraphStar
      \xxxSubParagraphNoStar
  }
  \newcommand{\xxxSubParagraphStar}[1]{\oldsubparagraph*{#1}\mbox{}}
  \newcommand{\xxxSubParagraphNoStar}[1]{\oldsubparagraph{#1}\mbox{}}
\fi
\makeatother

\usepackage{color}
\usepackage{fancyvrb}
\newcommand{\VerbBar}{|}
\newcommand{\VERB}{\Verb[commandchars=\\\{\}]}
\DefineVerbatimEnvironment{Highlighting}{Verbatim}{commandchars=\\\{\}}
% Add ',fontsize=\small' for more characters per line
\usepackage{framed}
\definecolor{shadecolor}{RGB}{241,243,245}
\newenvironment{Shaded}{\begin{snugshade}}{\end{snugshade}}
\newcommand{\AlertTok}[1]{\textcolor[rgb]{0.68,0.00,0.00}{#1}}
\newcommand{\AnnotationTok}[1]{\textcolor[rgb]{0.37,0.37,0.37}{#1}}
\newcommand{\AttributeTok}[1]{\textcolor[rgb]{0.40,0.45,0.13}{#1}}
\newcommand{\BaseNTok}[1]{\textcolor[rgb]{0.68,0.00,0.00}{#1}}
\newcommand{\BuiltInTok}[1]{\textcolor[rgb]{0.00,0.23,0.31}{#1}}
\newcommand{\CharTok}[1]{\textcolor[rgb]{0.13,0.47,0.30}{#1}}
\newcommand{\CommentTok}[1]{\textcolor[rgb]{0.37,0.37,0.37}{#1}}
\newcommand{\CommentVarTok}[1]{\textcolor[rgb]{0.37,0.37,0.37}{\textit{#1}}}
\newcommand{\ConstantTok}[1]{\textcolor[rgb]{0.56,0.35,0.01}{#1}}
\newcommand{\ControlFlowTok}[1]{\textcolor[rgb]{0.00,0.23,0.31}{\textbf{#1}}}
\newcommand{\DataTypeTok}[1]{\textcolor[rgb]{0.68,0.00,0.00}{#1}}
\newcommand{\DecValTok}[1]{\textcolor[rgb]{0.68,0.00,0.00}{#1}}
\newcommand{\DocumentationTok}[1]{\textcolor[rgb]{0.37,0.37,0.37}{\textit{#1}}}
\newcommand{\ErrorTok}[1]{\textcolor[rgb]{0.68,0.00,0.00}{#1}}
\newcommand{\ExtensionTok}[1]{\textcolor[rgb]{0.00,0.23,0.31}{#1}}
\newcommand{\FloatTok}[1]{\textcolor[rgb]{0.68,0.00,0.00}{#1}}
\newcommand{\FunctionTok}[1]{\textcolor[rgb]{0.28,0.35,0.67}{#1}}
\newcommand{\ImportTok}[1]{\textcolor[rgb]{0.00,0.46,0.62}{#1}}
\newcommand{\InformationTok}[1]{\textcolor[rgb]{0.37,0.37,0.37}{#1}}
\newcommand{\KeywordTok}[1]{\textcolor[rgb]{0.00,0.23,0.31}{\textbf{#1}}}
\newcommand{\NormalTok}[1]{\textcolor[rgb]{0.00,0.23,0.31}{#1}}
\newcommand{\OperatorTok}[1]{\textcolor[rgb]{0.37,0.37,0.37}{#1}}
\newcommand{\OtherTok}[1]{\textcolor[rgb]{0.00,0.23,0.31}{#1}}
\newcommand{\PreprocessorTok}[1]{\textcolor[rgb]{0.68,0.00,0.00}{#1}}
\newcommand{\RegionMarkerTok}[1]{\textcolor[rgb]{0.00,0.23,0.31}{#1}}
\newcommand{\SpecialCharTok}[1]{\textcolor[rgb]{0.37,0.37,0.37}{#1}}
\newcommand{\SpecialStringTok}[1]{\textcolor[rgb]{0.13,0.47,0.30}{#1}}
\newcommand{\StringTok}[1]{\textcolor[rgb]{0.13,0.47,0.30}{#1}}
\newcommand{\VariableTok}[1]{\textcolor[rgb]{0.07,0.07,0.07}{#1}}
\newcommand{\VerbatimStringTok}[1]{\textcolor[rgb]{0.13,0.47,0.30}{#1}}
\newcommand{\WarningTok}[1]{\textcolor[rgb]{0.37,0.37,0.37}{\textit{#1}}}

\usepackage{longtable,booktabs,array}
\usepackage{calc} % for calculating minipage widths
% Correct order of tables after \paragraph or \subparagraph
\usepackage{etoolbox}
\makeatletter
\patchcmd\longtable{\par}{\if@noskipsec\mbox{}\fi\par}{}{}
\makeatother
% Allow footnotes in longtable head/foot
\IfFileExists{footnotehyper.sty}{\usepackage{footnotehyper}}{\usepackage{footnote}}
\makesavenoteenv{longtable}
\usepackage{graphicx}
\makeatletter
\newsavebox\pandoc@box
\newcommand*\pandocbounded[1]{% scales image to fit in text height/width
  \sbox\pandoc@box{#1}%
  \Gscale@div\@tempa{\textheight}{\dimexpr\ht\pandoc@box+\dp\pandoc@box\relax}%
  \Gscale@div\@tempb{\linewidth}{\wd\pandoc@box}%
  \ifdim\@tempb\p@<\@tempa\p@\let\@tempa\@tempb\fi% select the smaller of both
  \ifdim\@tempa\p@<\p@\scalebox{\@tempa}{\usebox\pandoc@box}%
  \else\usebox{\pandoc@box}%
  \fi%
}
% Set default figure placement to htbp
\def\fps@figure{htbp}
\makeatother





\setlength{\emergencystretch}{3em} % prevent overfull lines

\providecommand{\tightlist}{%
  \setlength{\itemsep}{0pt}\setlength{\parskip}{0pt}}



 


\KOMAoption{captions}{tableheading}
\makeatletter
\@ifpackageloaded{caption}{}{\usepackage{caption}}
\AtBeginDocument{%
\ifdefined\contentsname
  \renewcommand*\contentsname{Table of contents}
\else
  \newcommand\contentsname{Table of contents}
\fi
\ifdefined\listfigurename
  \renewcommand*\listfigurename{List of Figures}
\else
  \newcommand\listfigurename{List of Figures}
\fi
\ifdefined\listtablename
  \renewcommand*\listtablename{List of Tables}
\else
  \newcommand\listtablename{List of Tables}
\fi
\ifdefined\figurename
  \renewcommand*\figurename{Figure}
\else
  \newcommand\figurename{Figure}
\fi
\ifdefined\tablename
  \renewcommand*\tablename{Table}
\else
  \newcommand\tablename{Table}
\fi
}
\@ifpackageloaded{float}{}{\usepackage{float}}
\floatstyle{ruled}
\@ifundefined{c@chapter}{\newfloat{codelisting}{h}{lop}}{\newfloat{codelisting}{h}{lop}[chapter]}
\floatname{codelisting}{Listing}
\newcommand*\listoflistings{\listof{codelisting}{List of Listings}}
\makeatother
\makeatletter
\makeatother
\makeatletter
\@ifpackageloaded{caption}{}{\usepackage{caption}}
\@ifpackageloaded{subcaption}{}{\usepackage{subcaption}}
\makeatother
\usepackage{bookmark}
\IfFileExists{xurl.sty}{\usepackage{xurl}}{} % add URL line breaks if available
\urlstyle{same}
\hypersetup{
  pdftitle={Lab 1},
  pdfauthor={Olivia Roberts},
  colorlinks=true,
  linkcolor={blue},
  filecolor={Maroon},
  citecolor={Blue},
  urlcolor={Blue},
  pdfcreator={LaTeX via pandoc}}


\title{Lab 1}
\author{Olivia Roberts}
\date{}
\begin{document}
\maketitle


\begin{Shaded}
\begin{Highlighting}[]
\FunctionTok{library}\NormalTok{(datasauRus)}
\end{Highlighting}
\end{Shaded}

\begin{verbatim}
Warning: package 'datasauRus' was built under R version 4.5.2
\end{verbatim}

\begin{Shaded}
\begin{Highlighting}[]
\FunctionTok{library}\NormalTok{(ggplot2)}
\end{Highlighting}
\end{Shaded}

\begin{verbatim}
Warning: package 'ggplot2' was built under R version 4.5.1
\end{verbatim}

\textbf{Question 1}

\begin{Shaded}
\begin{Highlighting}[]
\NormalTok{?datasaurus\_dozen}
\end{Highlighting}
\end{Shaded}

\begin{verbatim}
starting httpd help server ... done
\end{verbatim}

\begin{Shaded}
\begin{Highlighting}[]
\FunctionTok{table}\NormalTok{(datasaurus\_dozen}\SpecialCharTok{$}\NormalTok{dataset)}
\end{Highlighting}
\end{Shaded}

\begin{verbatim}

      away   bullseye     circle       dino       dots    h_lines high_lines 
       142        142        142        142        142        142        142 
slant_down   slant_up       star    v_lines wide_lines    x_shape 
       142        142        142        142        142        142 
\end{verbatim}

Answer: The datasaurus\_dozen file contains 1846 rows of 3 variables.
The 3 variables are a variable called ``dataset'' that indicates the
dataset the data is from, and x-values variable, and a y-values
variable. By accessing the dataset column, we can see a list of the 13
datasets inclued in datasaurus\_dozen and that they have 142 rows each.

\textbf{Question 2}

\begin{Shaded}
\begin{Highlighting}[]
\NormalTok{dino\_data }\OtherTok{\textless{}{-}}\NormalTok{ datasaurus\_dozen[datasaurus\_dozen}\SpecialCharTok{$}\NormalTok{dataset }\SpecialCharTok{==} \StringTok{\textquotesingle{}dino\textquotesingle{}}\NormalTok{,]}

\FunctionTok{ggplot}\NormalTok{(}\AttributeTok{data =}\NormalTok{ dino\_data, }\AttributeTok{mapping =} \FunctionTok{aes}\NormalTok{(}\AttributeTok{x =}\NormalTok{ x, }\AttributeTok{y =}\NormalTok{ y)) }\SpecialCharTok{+}
  \FunctionTok{geom\_point}\NormalTok{()}
\end{Highlighting}
\end{Shaded}

\pandocbounded{\includegraphics[keepaspectratio]{lab-01_files/figure-pdf/unnamed-chunk-3-1.pdf}}

\begin{Shaded}
\begin{Highlighting}[]
\FunctionTok{cor}\NormalTok{(dino\_data}\SpecialCharTok{$}\NormalTok{x, dino\_data}\SpecialCharTok{$}\NormalTok{y)}
\end{Highlighting}
\end{Shaded}

\begin{verbatim}
[1] -0.06447185
\end{verbatim}

\textbf{Question 3}

\begin{Shaded}
\begin{Highlighting}[]
\NormalTok{star\_data }\OtherTok{\textless{}{-}}\NormalTok{ datasaurus\_dozen[datasaurus\_dozen}\SpecialCharTok{$}\NormalTok{dataset }\SpecialCharTok{==} \StringTok{\textquotesingle{}star\textquotesingle{}}\NormalTok{,]}

\FunctionTok{ggplot}\NormalTok{(}\AttributeTok{data =}\NormalTok{ star\_data, }\AttributeTok{mapping =} \FunctionTok{aes}\NormalTok{(}\AttributeTok{x =}\NormalTok{ x, }\AttributeTok{y =}\NormalTok{ y)) }\SpecialCharTok{+}
  \FunctionTok{geom\_point}\NormalTok{()}
\end{Highlighting}
\end{Shaded}

\pandocbounded{\includegraphics[keepaspectratio]{lab-01_files/figure-pdf/unnamed-chunk-4-1.pdf}}

\begin{Shaded}
\begin{Highlighting}[]
\FunctionTok{cor}\NormalTok{(star\_data}\SpecialCharTok{$}\NormalTok{x, star\_data}\SpecialCharTok{$}\NormalTok{y)}
\end{Highlighting}
\end{Shaded}

\begin{verbatim}
[1] -0.0629611
\end{verbatim}

Answer: The correlation coefficient for the dino and star datasets are
very similar; they are -0.064 and -0.063 respectively. They are both
negative, so in both datasets, as x increases, y decreases. The dino
dataset has a slightly stronger negative correlation, but neither of the
datasets have a very strong negative correlation since the points are
very scattered and the magnitude of the correlation is not very large.
Despite having similar correlations, the two datasets look very
different.

\textbf{Question 4}

\begin{Shaded}
\begin{Highlighting}[]
\NormalTok{circle\_data }\OtherTok{\textless{}{-}}\NormalTok{ datasaurus\_dozen[datasaurus\_dozen}\SpecialCharTok{$}\NormalTok{dataset }\SpecialCharTok{==} \StringTok{\textquotesingle{}circle\textquotesingle{}}\NormalTok{,]}

\FunctionTok{ggplot}\NormalTok{(}\AttributeTok{data =}\NormalTok{ circle\_data, }\AttributeTok{mapping =} \FunctionTok{aes}\NormalTok{(}\AttributeTok{x =}\NormalTok{ x, }\AttributeTok{y =}\NormalTok{ y)) }\SpecialCharTok{+}
  \FunctionTok{geom\_point}\NormalTok{()}
\end{Highlighting}
\end{Shaded}

\pandocbounded{\includegraphics[keepaspectratio]{lab-01_files/figure-pdf/unnamed-chunk-5-1.pdf}}

\begin{Shaded}
\begin{Highlighting}[]
\FunctionTok{cor}\NormalTok{(circle\_data}\SpecialCharTok{$}\NormalTok{x, circle\_data}\SpecialCharTok{$}\NormalTok{y)}
\end{Highlighting}
\end{Shaded}

\begin{verbatim}
[1] -0.06834336
\end{verbatim}

Answer: The correlation coefficient of the dino dataset is -0.064, while
the correlation coefficient for the circle dataset is -0.068. Again,
both datasets have a negative correlation (as x increases, y decreases
overall), but the circle dataset now has a slightly stronger magnitude
of correlation. However, both datasets still have a very similar
magnitude of correlation despite looking very different.

\textbf{Question 5}

\begin{Shaded}
\begin{Highlighting}[]
\FunctionTok{ggplot}\NormalTok{(datasaurus\_dozen, }\FunctionTok{aes}\NormalTok{(}\AttributeTok{x =}\NormalTok{ x, }\AttributeTok{y =}\NormalTok{ y, }\AttributeTok{color =}\NormalTok{ dataset))}\SpecialCharTok{+}
  \FunctionTok{geom\_point}\NormalTok{()}\SpecialCharTok{+}
  \FunctionTok{facet\_wrap}\NormalTok{(}\SpecialCharTok{\textasciitilde{}}\NormalTok{ dataset, }\AttributeTok{ncol =} \DecValTok{3}\NormalTok{) }\SpecialCharTok{+}
  \FunctionTok{theme}\NormalTok{(}\AttributeTok{legend.position =} \StringTok{"none"}\NormalTok{)}
\end{Highlighting}
\end{Shaded}

\pandocbounded{\includegraphics[keepaspectratio]{lab-01_files/figure-pdf/unnamed-chunk-6-1.pdf}}

\textbf{Question 6}

\begin{Shaded}
\begin{Highlighting}[]
\FunctionTok{sapply}\NormalTok{(}\FunctionTok{unique}\NormalTok{(datasaurus\_dozen}\SpecialCharTok{$}\NormalTok{dataset), }\ControlFlowTok{function}\NormalTok{(name)\{}
\NormalTok{    subset }\OtherTok{\textless{}{-}}\NormalTok{ datasaurus\_dozen[datasaurus\_dozen}\SpecialCharTok{$}\NormalTok{dataset }\SpecialCharTok{==}\NormalTok{ name, ]}
    \FunctionTok{return}\NormalTok{(}\FunctionTok{cor}\NormalTok{(subset}\SpecialCharTok{$}\NormalTok{x, subset}\SpecialCharTok{$}\NormalTok{y))}
\NormalTok{\})}
\end{Highlighting}
\end{Shaded}

\begin{verbatim}
       dino        away     h_lines     v_lines     x_shape        star 
-0.06447185 -0.06412835 -0.06171484 -0.06944557 -0.06558334 -0.06296110 
 high_lines        dots      circle    bullseye    slant_up  slant_down 
-0.06850422 -0.06034144 -0.06834336 -0.06858639 -0.06860921 -0.06897974 
 wide_lines 
-0.06657523 
\end{verbatim}




\end{document}
